% #region imports
\documentclass[10pt, twoside]{article}          % document type
\usepackage[brazil]{babel}                      % portuguese support
\usepackage[utf8]{inputenc}                     % almost every symbol
\usepackage[T1]{fontenc}                        % font support

\PassOptionsToPackage{table}{xcolor}            % table colors

\usepackage{amsmath}                            % math enhancements
\usepackage{amssymb}                            % math symbols
\usepackage{amsthm}                             % math proof
\usepackage{bera}                               % mono font
\usepackage{booktabs}                           % figure tables
\usepackage{cancel}                             % math cancel
\usepackage{color}                              % page and text color
\usepackage{dsfont}                             % indicator function
\usepackage{enumitem}                           % list settings
\usepackage{etoolbox}                           % bordermatrix patch
\usepackage{float}                              % figure here
\usepackage{fourier}                            % more readable math
\usepackage{gensymb}                            % math degree
\usepackage{geometry}                           % page settings
\usepackage{graphicx}                           % images
\usepackage{helvet}                             % Arial font
\usepackage{hyperref}                           % links
\usepackage{listings}                           % code environment
\usepackage{mathtools}                          % mathclap
\usepackage{multicol}                           % page columns
% \usepackage{newtxmath}                          % more readable math
\usepackage{nicefrac}                           % nicefrac horizontal fraction
\usepackage{pifont}                             % correct and wrong symbols
\usepackage[fontsize=8pt]{scrextend}            % global font size
\usepackage{tabto}                              % tabto positioning
\usepackage{textcomp}                           % text leaf
\usepackage{tikz}                               % graphics
% \usepackage[table]{xcolor}                      % table colors (option clash)
\usepackage{xcolor}                             % html colors
% #endregion

% #region settings

\DeclareSymbolFont{AMSb}{U}{msb}{m}{n}
\makeatletter
\DeclareSymbolFontAlphabet{\math@bb}{AMSb}
\AtBeginDocument{\protected\def\mathbb{\math@bb}} 
\makeatother

\DeclareMathAlphabet{\mathcal}{OMS}{cmsy}{m}{n}

\DeclareMathAlphabet{\mathfrak}{U}{jkpmia}{m}{it}
\SetMathAlphabet{\mathfrak}{bold}{U}{jkpmia}{bx}{it}


\usetikzlibrary{arrows, positioning, shapes}    % state diagram

\geometry{                                      % page settings
  papersize={216mm,279mm}, landscape,
  top=1.525cm,  outer=1.125cm,  bottom=1.125cm, inner=1.225cm,
  includefoot,  footskip=0.5cm
}

\geometry{papersize={210mm,297mm}}              % A4 paper

% \pagecolor{black}                               % page background color
% \color{white}                                   % text default color

\renewcommand{\familydefault}{\sfdefault}       % Arial font

\lstset{
  literate=
  {á}{{\'a}}1 {à}{{\à}}1 {ã}{{\~a}}1
  {é}{{\'e}}1 {ê}{{\^e}}1
  {í}{{\'i}}1
  {ó}{{\'o}}1 {õ}{{\~o}}1
  {ú}{{\'u}}1 {ü}{{\"u}}1
  {ç}{{\c{c}}}1
}                                               % accents in code

% \delimitershortfall=-1pt                        % bigger wrapping brackets
\delimitershortfall=1pt                         % average wrapping brackets
\def\dis{\displaystyle}                         % big inline math

\makeatletter

  \def\resetMathstrut@{%                        % default \left and \right in math mode (1/2)
    \setbox\z@\hbox{%
      \mathchardef\@tempa\mathcode`\[\relax
      \def\@tempb##1"##2##3{\the\textfont"##3\char"}%
      \expandafter\@tempb\meaning\@tempa \relax
    }%
    \ht\Mathstrutbox@\ht\z@ \dp\Mathstrutbox@\dp\z@}
  \mathchardef\@tempa\mathcode`\]\relax

  \def\cantox@vector#1#2#3#4#5#6#7#8{%
    \dimen@.5\p@
    \setbox\z@\vbox{\boxmaxdepth.5\p@
    \hbox{\kern-1.2\p@\kern#1\dimen@$#7{#8}\m@th$}}%
    \ifx\canto@fil\hidewidth  \wd\z@\z@ \else \kern-#6\unitlength \fi
    \ooalign{%
      \canto@fil$\m@th \CancelColor
      \vcenter{\hbox{\dimen@#6\unitlength \kern\dimen@
        \multiply\dimen@#4\divide\dimen@#3 \vrule\@depth\dimen@\@width\z@
        \vector(#3,-#4){#5}%
      }}_{\raise-#2\dimen@\copy\z@\kern-\scriptspace}$%
      \canto@fil \cr
      \hfil \box\@tempboxa \kern\wd\z@ \hfil \cr}}
  \def\bcancelto#1#2{\let\canto@vector\cantox@vector\cancelto{#1}{#2}}

  \newlength{\normalparindent}
  \AtBeginDocument{\setlength{\normalparindent}{\parindent}}
\makeatother

\let\oldphantom\vphantom
\let\vphantom\relax
\begingroup                                     % default \left and \right in math mode (1/2)
  \catcode`(\active \xdef({\mathopen{}\left\string(\vphantom{1j}}
  \catcode`)\active \xdef){\right\string)\mathclose{}}
\endgroup
\mathcode`(="8000 \mathcode`)="8000
\let\vphantom\oldphantom

\DeclareMathSymbol{*}{\mathbin}{symbols}{"01}   % default \cdot in math mode
% #endregion

% #region commands
\newcommand*\diff{\mathop{}\!\mathrm{d}}        % differential d

\DeclareMathOperator{\dom}{dom}                 % dom function
\DeclareMathOperator{\img}{img}                 % img function
\DeclareMathOperator{\mdc}{mdc}                 % mdc function
\DeclareMathOperator{\adj}{adj}                 % adj function
\DeclareMathOperator{\proj}{proj}               % projection function

\newcommand{\Reals}{\mathds{R}}                 % Reals set
\newcommand{\Ints}{\mathds{Z}}                  % Integers set
\newcommand{\Nats}{\mathds{N}}                  % Naturals set
\newcommand{\Rats}{\mathds{Q}}                  % Rationals set
\newcommand{\Irats}{\mathds{I}}                 % Irrationals set
\newcommand{\Primes}{\mathds{P}}                % Primes set
\newcommand{\ind}{\mathds{1}}                   % indicator function

\newcommand{\given}{\,\middle|\,}               % conditional probability

\renewcommand{\complement}{\mathsf{c}}          % set complement

\DeclareRobustCommand{\Omicron}{%
  \text{\small\usefont{OMS}{cmsy}{m}{n}O}%
}                                               % big omicron
\DeclareRobustCommand{\omicron}{%
  \text{\small\usefont{OMS}{cmsy}{m}{n}o}%
}                                               % small omicron

\newcommand{\bigO}{\Omicron}                    % big O
\newcommand{\smallo}{\omicron}                  % small o

\newcommand{\defeq}{\vcentcolon=}               % definition
\newcommand{\eqdef}{=\vcentcolon}               % reverse definition

\newcommand{\cmark}{\ding{51}}                  % correct symbol
\newcommand{\xmark}{\ding{55}}                  % wrong symbol
\newcommand{\cross}{\ding{61}}                  % amortized symbol

\renewcommand{\deg}{\!\degree\:}                % math degree

\renewcommand{\binom}[2]{\l({{#1}\atop#2}\r)}   % combinations

\DeclareRobustCommand{\Chi}
  {{\mathpalette\irchi\relax}}
\newcommand{\irchi}[2]
  {\raisebox{\depth}{$#1\chi$}}                 % uppercase chi

\newcommand{\overtext}[2]
  {\overset{\mathclap{\text{#1}}}{#2}}          % text over
\newcommand{\undertext}[2]
  {\underset{\mathclap{\text{#1}}}{#2}}         % text under
\newcommand{\overmath}[2]
  {\overset{\mathclap{#1}}{#2}}                 % math over
\newcommand{\undermath}[2]
  {\underset{\mathclap{#1}}{#2}}                % math under
\newcommand{\canceltext}[2]
  {\smash{\overtext{#1}{\cancel{#2}}}}          % text cancel

\newcommand{\vghost}[1]{{%                      % artificial math size
  \delimitershortfall=-1pt
  \vphantom{
    \begingroup
    \lccode`m=`(\relax
    \lowercase\expandafter{\romannumeral#1000}%
    \lccode`m=`)\relax
    \lowercase\expandafter{\romannumeral#1000}%
    \endgroup
  }
}}

\renewcommand{\l}{\mathopen{}\left}             % \left alias
\renewcommand{\r}{\vphantom{1j}\right}          % \right alias
% \newcommand{\ls}[1]
%   {\mathopen{}\noexpand\begingroup\left#1}      % \left alias
% \newcommand{\rs}[1]
%   {\vphantom{1j}\endgroup\right#1}              % \right alias

\let\oldtextleaf\textleaf
\renewcommand{\textleaf}
  {{\fontfamily{cmr}\selectfont \oldtextleaf}}  % textleaf

\renewcommand{\qed}{\hfill$\blacksquare$}       % Black-filled qed

\newcommand{\x}[1]{\discretionary{#1}{#1}{#1}}  % correct hyphenation
\newcommand{\y}{\hspace{0pt}}                   % breackable non-space    

\newcommand{\triple}[4]{%
  \parbox{.333#4}{#1\hfill}%
  \parbox{.333#4}{\hfil#2\hfil}%
  \parbox{.333#4}{\hfill#3}%
}                                               % triple align
% #endregion

% #region customizations
\newenvironment{proof*}[1][proof*]              % better proof environment
  {\proof[#1]\vspace{0.5em}\vspace*{-\baselineskip}
  \hspace{\parindent}\leftskip=.5cm\rightskip=.5cm}
  % {\vspace*{-0.5\baselineskip}\rightskip=0cm\endproof}
  {\vspace*{-1.5\baselineskip}
  
  \rightskip=0cm\endproof}

% \let\bbordermatrix\bordermatrix
\patchcmd{\bordermatrix}{8.75}{4.75}{}{}
\patchcmd{\bordermatrix}{\left(}{\left[}{}{}    % bordermatrix with angular brackets (left)
\patchcmd{\bordermatrix}{\right)}{\right]}{}{}  % bordermatrix with angular brackets (right)
\patchcmd{\bordermatrix}
  {\begingroup}{\begingroup\openup1\jot}{}{}    % bordermatrix col height

\hypersetup{
  colorlinks,
  linkcolor={red!50!black},
  citecolor={blue!50!black},
  urlcolor={blue!80!black}
}

\lstset{
  basicstyle=\ttfamily,
  escapeinside={(*@}{@*)},
  mathescape=true,
  extendedchars=true,
  inputencoding=utf8
}                                               % styles in listings

\DeclareFixedFont{\ttb}{T1}{txtt}{bx}{n}{8}     % for bold monofont
\DeclareFixedFont{\ttm}{T1}{txtt}{m}{n}{8}      % for normal monfont

% \definecolor{deepblue}{rgb}{0,0,0.5}
\newcommand{\pythonstyle}{\lstset{              % python code style
  xleftmargin=\dimexpr.5cm+\parindent\relax,
  language=Python,
  % basicstyle=\ttm,
  otherkeywords={self},
  % keywordstyle=\ttb\color{purple},
  keywordstyle={\bfseries},
  % emph={MyClass,__init__},
  % emphstyle=\ttb\color{deepred},
  % stringstyle=\color{deepgreen},
  % frame=tb,
  showstringspaces=false
}}

\lstnewenvironment{python}[1][]
  {\pythonstyle\lstset{#1}}{}                   % python environment

\newcommand\pythonexternal[2][]
  {{\pythonstyle\lstinputlisting[#1]{#2}}}      % external python code

\newcommand\pythoninline[1]
  {{\pythonstyle\lstinline!#1!}}                % inline python code

\renewcommand{\arraystretch}{1.2}               % table vertical padding

\renewcommand{\arraystretch}{1.2}               % table vertical padding

\lstnewenvironment{pseudocode}[1][]             % pseudocode environment
  {\lstset{
    xleftmargin=\dimexpr.5cm+\parindent\relax,
    gobble=6,
    #1,
    emph={
      if, else, elif, and, or, not,
      for, while, continue, break, return, yield, do, to,
      true, false, null
    },
    emphstyle={\bfseries}
  }}{}

\setlist[enumerate, 1]                          % default level 1 list
  {wide, label=\bfseries\arabic*., labelwidth=10pt, labelindent=0pt}
\setlist[enumerate, 2]                          % default level 2 list
  {wide, label=\bfseries(\alph*), topsep=0pt, labelwidth=10pt, labelindent=\leftskip, leftmargin=0pt}
\setlist[enumerate, 3]                          % default level 2 list
  {wide, label=\bfseries\roman*, topsep=0pt, labelwidth=10pt, labelindent=\leftskip, leftmargin=0pt}

\newenvironment{enumerate*}[1][,]               % text enumerate
  {\begin{enumerate}[
    itemindent=\leftskip+\parindent, labelindent=\leftskip+\parindent, 
    wide, topsep=0pt,
    label={\bfseries\arabic*.}, labelwidth=10pt, labelindent=\leftskip+\parindent, 
    leftmargin=\leftskip, rightmargin=\rightskip,
    #1
  ]}
  {\end{enumerate}}

\newenvironment{itemize*}[1][,]                 % text itemize
  {\begin{itemize}[
    itemindent=\leftskip+\parindent, labelindent=\leftskip+\parindent, 
    wide, topsep=0pt,
    label={\raisebox{-0.5mm}{\scalebox{1.5}{$\bullet$}}}, labelwidth=10pt, labelindent=\leftskip+\parindent, 
    leftmargin=\leftskip, rightmargin=\rightskip,
    #1
  ]}
  {\end{itemize}}
% #endregion

\begin{document}
\begin{multicols*}{2}
\setlength{\columnseprule}{0.4pt}

\begin{center}
  \textbf{\large MAC0444 --- Sistemas Baseados em Conhecimento}

  \textit{Departamento de Ciência da Computação}

  \bigskip
  \triple{\textbf{Prof.ª} Renata Wassermann}{\textbf{Lista 3}}{16 de outubro de 2019}{\columnwidth}

  \bigskip
  {\bf Aluno} Vitor Santa Rosa Gomes, 10258862, vitorssrg@usp.br

  {\bf Curso} Bacharelado em Ciência da Computação, IME-USP
\end{center}

\begin{enumerate}
  \renewcommand{\b}{\overline}
  \newcommand{\s}{\smash}
  \newcommand{\lxor}{\mathbin{\,\underline{\!\vee\!}\,}}

  \item[\textbf{1.}] João é cardiologista. Sua irmã, Marta, está desempregada. O pai deles dois, 
  Pedro, é casado com Olívia, uma geriatra. Pedro é arqueologista e tem dois filhos no total.
  \begin{proof*}[\unskip\nopunct]
    \begin{enumerate}
      \item Explique o que significa o conceito abaixo:
      \[
              \texttt{Human}
        \sqcap\lnot\texttt{Female}
        \sqcap(\exists \texttt{married}.\texttt{Doctor})
        \sqcap(\forall \texttt{hasChild}.(\texttt{Doctor}\sqcup \texttt{Professor}))
      \]
      
      em que os conceitos e papéis significam:
      \begin{itemize*}
        \item $\texttt{Human}$: é humano(a)
        \item $\texttt{Female}$: é do sexo feminino
        \item $\texttt{Doctor}$: é médico(a)
        \item $\texttt{Professor}$: é professor(a)
        \item $\texttt{hasChild}(x,\ y)$: $x$ tem $y$ como filho(a)
        \item $\texttt{married}(x,\ y)$: $x$ é casado(a) com $y$
      \end{itemize*}

        O conceito é um humano que não é do sexo feminino, que é casado com algum médico, e cujos 
        filhos são médicos ou professores.
      
      \item Considerando apenas as quatro pessoas mencionadas, João, Marta, Pedro e Olívia, podemos 
      afirmar que algum deles pertence a esse conceito? Justifique.

        Mesmo assumindo que todos sejam humanos, não é possível, já que
        \begin{enumerate*}
          \item João não é casado com alguém,
          \item Marta não é casada com alguém,
          \item Pedro tem como filha Marta, que é nem médica nem professora, e
          \item Olívia não é casada com algum médico.
        \end{enumerate*}

      \item Se uma das 4 pessoas acima não existisse, sua resposta mudaria? Como e por quê?
      
        Sem Marta, Pedro pertenceria ao conceito:
        \begin{enumerate*}
          \item assume-se Pedro humano (razoável);
          \item assume-se Pedro do sexo não feminino (razoável);
          \item Pedro é casado com algum(a) médico(a), Olívia;
          \item O(a) único(a) filho(a) de Pedro, João, é médico.
        \end{enumerate*}
    \end{enumerate}
  \end{proof*}

  \vfill\null\columnbreak
  \item[\textbf{2.}] Considere a seguinte $\mathcal{T}$-Box $\mathcal{T}$:
  \begin{align*}
    \texttt{Mulher} &\sqsubseteq \texttt{Pessoa}       \\
    \texttt{Homem}  &\sqsubseteq \texttt{Pessoa}       \\
    \texttt{Homem}  &\sqsubseteq \lnot\texttt{Mulher}  \\
    \texttt{Mulher} &\sqsubseteq \lnot\texttt{Homem}   \\
  \end{align*} 

  Verifique se o seguinte axioma é consequência lógica de $\mathcal{T}$. Dê uma demonstração 
  semântica ou um contra-exemplo.
  \[
    \texttt{Pessoa} \sqcap \lnot\texttt{Homem} \equiv \texttt{Mulher}
  \]
  \begin{proof*}[\unskip\nopunct]
    Não é consequência lógica: basta $x$ \texttt{Pessoa} ser nem \texttt{Mulher} nem \texttt{Homem}. 
    contraexemplo:
    \begin{align*}
      \Delta^\mathcal{I}          &= \l\{x,\ y\r\}  \\
      \texttt{Pessoa}^\mathcal{I} &= \l\{x,\ y\r\}  \\
      \texttt{Mulher}^\mathcal{I} &= \l\{x\r\}      \\
      \texttt{Homem}^\mathcal{I}  &= \l\{\r\}       \\
    \end{align*}

    Pois
    \[
                          \texttt{Pessoa} \sqcap \lnot\texttt{Homem} 
                  \equiv  \l\{x,\ y\r\} 
      \not\equiv 
                          \l\{x\r\}
                  \equiv  \texttt{Mulher}
    \]
  \end{proof*}

  \item[\textbf{3.}] Traduza o seguinte axioma em uma sentença equivalente em lógica de primeira
  ordem:
  \[
    \texttt{PaiDeMedicos}
      \sqsubseteq 
        \exists\texttt{temFilho}(\texttt{Homem} \sqcup \texttt{Mulher}) 
        \sqcap\forall\texttt{temFilho}(\texttt{Medico})
  \]
  \begin{proof*}[\unskip\nopunct]
    \delimitershortfall=-1pt
    \begin{align*}
      \underbrace{
        \underbrace{\texttt{PaiDeMedicos}}_{\texttt{PaiDeMedicos}(x)}
      \sqsubseteq 
        \underbrace{
          \underbrace{\exists
            \underbrace{\texttt{temFilho}}_{\texttt{temFilho}(x,\ y)}(\vghost{1}\r.
              \underbrace{
                \underbrace{\texttt{Homem}}_{\texttt{Homem}(x)}
              \sqcup 
                \underbrace{\texttt{Mulher}}_{\texttt{Mulher}(x)}
              }_{\texttt{Homem}(x)\lor\texttt{Mulher}(x)}
            \l.\vghost{1}) 
          }_{\exists y(\texttt{temFilho}(x,\ y)\land(\texttt{Homem}(y)\lor\texttt{Mulher}(y)))}
        \sqcap
          \underbrace{\forall
            \underbrace{\texttt{temFilho}}_{\texttt{temFilho}(x,\ y)}(\vghost{1}\r.
              \underbrace{\texttt{Medico}}_{\texttt{Medico}(x)}
            \l.\vghost{1})
          }_{\forall y(\texttt{temFilho}(x,\ y)\rightarrow\texttt{Medico}(y))}
        }_{     \exists y(\texttt{temFilho}(x,\ y)\land(\texttt{Homem}(y)\lor\texttt{Mulher}(y)))
           \land\forall y(\texttt{temFilho}(x,\ y)\rightarrow\texttt{Medico}(y))}
      }_{\forall x(           \texttt{PaiDeMedicos}(x)
                   \rightarrow     \exists y(\texttt{temFilho}(x,\ y)\land(\texttt{Homem}(y)\lor\texttt{Mulher}(y)))
                              \land\forall y(\texttt{temFilho}(x,\ y)\rightarrow\texttt{Medico}(y)))}
    \end{align*}
  \end{proof*}

  \vfill\null\columnbreak
  \item[\textbf{4.}] Sejam os seguintes conceitos:
  \begin{align*}
    \texttt{Vegano}       &\equiv \texttt{Homem} \sqcap \forall\texttt{come}.\texttt{Planta}        \\
    \texttt{Vegetariano}  &\equiv         (\texttt{Homem} \sqcup \texttt{Mulher}) 
                                  \sqcap  \forall\texttt{come}.(       \texttt{Planta} 
                                                                \sqcup \texttt{Laticinio})          \\
  \end{align*} 

  Mostre, \textbf{utilizando tableaux}, que $\texttt{Vegano}\sqsubseteq\texttt{Vegetariano}$.
  \begin{proof*}[\unskip\nopunct]
    \delimitershortfall=-1pt
    Por ser um método de refutação, vamos mostrar que, de forma equivalente, 
    $\lnot(\texttt{Vegano}\sqsubseteq\texttt{Vegetariano})$ é insatisfazível.

    Em seguida, substituindo pelas definições dadas e propagando a negação:
    \[
              (
                      \texttt{Homem}
              \sqcap
                      \forall\texttt{come}.\texttt{Planta}
              )
      \sqcap
              (
                      (\lnot\texttt{Homem}\sqcap\lnot\texttt{Mulher})
              \sqcup
                      \exists\texttt{come}.(\lnot\texttt{Planta}\sqcap\lnot\texttt{Laticinio})
              )
    \]

    Aplicando o tableaux, vemos que $x\in\lnot(\texttt{Vegano}\sqsubseteq\texttt{Vegetariano})$ é 
    uma contradição.
    \begin{center}
      \footnotesize
      \begin{tikzpicture}[
        auto, thick,
        node/.style={draw, circle, thick, text centered, 
                     minimum height=0.50cm, minimum width=0.50cm},
        star/.style={draw, diamond, thick, text centered, 
                     minimum height=0.50cm, minimum width=0.50cm},
        block/.style={draw, thick, text centered, 
                      minimum height=0.50cm, minimum width=0.50cm},
        title/.style={draw=none, text centered, 
                      minimum height=0.50cm, minimum width=0.50cm},
        every loop/.style={}
      ]

        \node[title]  (x00) [                   xshift=-5.50cm]   {\textbf{1.}};
        \node[title]  (x01)                                       {$((\texttt{Homem}\sqcap\forall\texttt{come}.\texttt{Planta})\sqcap((\lnot\texttt{Homem}\sqcap\lnot\texttt{Mulher})\sqcup\exists\texttt{come}.(\lnot\texttt{Planta}\sqcap\lnot\texttt{Laticinio})))(x)$};


        \node[title]  (x10) [below=0.50 of x01, xshift=-5.50cm]   {\textbf{2.}};
        \node[title]  (x11) [below=0.50 of x01                ]   {$(\texttt{Homem}\sqcap\forall\texttt{come}.\texttt{Planta})(x)$};
        \node[title]  (x12) [below=0.50 of x01, xshift= 5.50cm]   {\textbf{1.}$\sqcap$};

        \node[title]  (x20) [below=0.50 of x11, xshift=-5.50cm]   {\textbf{3.}};
        \node[title]  (x21) [below=0.50 of x11                ]   {$((\lnot\texttt{Homem}\sqcap\lnot\texttt{Mulher})\sqcup\exists\texttt{come}.(\lnot\texttt{Planta}\sqcap\lnot\texttt{Laticinio}))(x)$};
        \node[title]  (x22) [below=0.50 of x11, xshift= 5.50cm]   {\textbf{1.}$\sqcap$};

        \node[title]  (x30) [below=0.50 of x21, xshift=-5.50cm]   {\textbf{4.}};
        \node[title]  (x31) [below=0.50 of x21                ]   {$\texttt{Homem}(x)$};
        \node[title]  (x32) [below=0.50 of x21, xshift= 5.50cm]   {\textbf{2.}$\sqcap$};

        \node[title]  (x40) [below=0.50 of x31, xshift=-5.50cm]   {\textbf{5.}};
        \node[title]  (x41) [below=0.50 of x31                ]   {$(\forall\texttt{come}.\texttt{Planta})(x)$};
        \node[title]  (x42) [below=0.50 of x31, xshift= 5.50cm]   {\textbf{2.}$\sqcap$};

        \node[title]  (x50) [below=0.50 of x41, xshift=-5.50cm]   {\textbf{6.}};
        \node[title]  (x51) [below=0.50 of x41, xshift=-3.00cm]   {$((\lnot\texttt{Homem}\sqcap\lnot\texttt{Mulher}))(x)$};
        \node[title]  (x52) [below=0.50 of x41, xshift=-0.50cm]   {\textbf{3.}$\sqcup$};

        \node[title]  (x53) [below=0.50 of x41, xshift= 0.50cm]   {\textbf{7.}};
        \node[title]  (x54) [below=0.50 of x41, xshift= 3.00cm]   {$(\exists\texttt{come}.(\lnot\texttt{Planta}\sqcap\lnot\texttt{Laticinio}))(x)$};
        \node[title]  (x55) [below=0.50 of x41, xshift= 5.50cm]   {\textbf{3.}$\sqcup$};

        \node[title]  (x60) [below=0.50 of x51, xshift=-2.50cm]   {\textbf{8.}};
        \node[title]  (x61) [below=0.50 of x51                ]   {$\lnot\texttt{Homem}(x)$};
        \node[title]  (x62) [below=0.50 of x51, xshift= 2.50cm]   {\textbf{3.}$\sqcap$};


        \node[title]  (x71) [below=0.50 of x61                ]   {$\textbf{4.}\land\textbf{8.}\equiv\bot$};


        \node[title]  (x63) [below=0.50 of x54, xshift=-2.50cm]   {\textbf{9.}};
        \node[title]  (x64) [below=0.50 of x54                ]   {$\texttt{come}(x,\ y)$};
        \node[title]  (x65) [below=0.50 of x54, xshift= 2.50cm]   {\textbf{7.}$\exists$};

        \node[title]  (x73) [below=0.50 of x64, xshift=-2.50cm]   {\textbf{10.}};
        \node[title]  (x74) [below=0.50 of x64                ]   {$(\lnot\texttt{Planta}\sqcap\lnot\texttt{Laticinio})(y)$};
        \node[title]  (x75) [below=0.50 of x64, xshift= 2.50cm]   {\textbf{7.}$\exists$};

        \node[title]  (x83) [below=0.50 of x74, xshift=-2.50cm]   {\textbf{11.}};
        \node[title]  (x84) [below=0.50 of x74                ]   {$\texttt{Planta}(y)$};
        \node[title]  (x85) [below=0.50 of x74, xshift= 2.50cm]   {$\textbf{5.}\land\textbf{9.}\forall$};

        \node[title]  (x93) [below=0.50 of x84, xshift=-2.50cm]   {\textbf{12.}};
        \node[title]  (x94) [below=0.50 of x84                ]   {$\lnot\texttt{Planta}(y)$};
        \node[title]  (x95) [below=0.50 of x84, xshift= 2.50cm]   {\textbf{10.}$\sqcap$};


        \node[title]  (xA4) [below=0.50 of x94                ]   {$\textbf{11.}\land\textbf{12.}\equiv\bot$};
        

        \path[->]
          (x01) edge                              (x11)
          (x11) edge                              (x21)
          (x21) edge                              (x31)
          (x31) edge                              (x41)
            (x41) edge                              (x51)
            (x51) edge                              (x61)
            (x61) edge                              (x71)

            (x41) edge                              (x54)
            (x54) edge                              (x64)
            (x64) edge                              (x74)
            (x74) edge                              (x84)
            (x84) edge                              (x94)
            (x94) edge                              (xA4)
        ;
      \end{tikzpicture}
    \end{center}
  \end{proof*}

\end{enumerate}

\vfill\null
\vfill\null
{
  \raggedleft
  \tiny
  \it

  You held your head like a hero

  On a history book page

  It was the end of a decade

  But the start of an age

}
\end{multicols*}
\end{document}