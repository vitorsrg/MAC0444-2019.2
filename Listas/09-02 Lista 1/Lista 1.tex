% #region imports
\documentclass[10pt, twoside]{article}          % document type
\usepackage[brazil]{babel}                      % portuguese support
\usepackage[utf8]{inputenc}                     % almost every symbol
\usepackage[T1]{fontenc}                        % font support

\PassOptionsToPackage{table}{xcolor}            % table colors

\usepackage{amsmath}                            % math enhancements
\usepackage{amssymb}                            % math symbols
\usepackage{amsthm}                             % math proof
\usepackage{bera}                               % mono font
\usepackage{booktabs}                           % figure tables
\usepackage{cancel}                             % math cancel
\usepackage{color}                              % page and text color
\usepackage{dsfont}                             % indicator function
\usepackage{enumitem}                           % list settings
\usepackage{etoolbox}                           % bordermatrix patch
\usepackage{float}                              % figure here
\usepackage{fourier}                            % more readable math
\usepackage{gensymb}                            % math degree
\usepackage{geometry}                           % page settings
\usepackage{graphicx}                           % images
\usepackage{helvet}                             % Arial font
\usepackage{hyperref}                           % links
\usepackage{listings}                           % code environment
\usepackage{mathtools}                          % mathclap
\usepackage{multicol}                           % page columns
% \usepackage{newtxmath}                          % more readable math
\usepackage{nicefrac}                           % nicefrac horizontal fraction
\usepackage{pifont}                             % correct and wrong symbols
\usepackage[fontsize=8pt]{scrextend}            % global font size
\usepackage{tabto}                              % tabto positioning
\usepackage{textcomp}                           % text leaf
\usepackage{tikz}                               % graphics
% \usepackage[table]{xcolor}                      % table colors (option clash)
\usepackage{xcolor}                             % html colors
% #endregion

% #region settings

\DeclareSymbolFont{AMSb}{U}{msb}{m}{n}
\makeatletter
\DeclareSymbolFontAlphabet{\math@bb}{AMSb}
\AtBeginDocument{\protected\def\mathbb{\math@bb}} 
\makeatother

\DeclareMathAlphabet{\mathcal}{OMS}{cmsy}{m}{n}

\DeclareMathAlphabet{\mathfrak}{U}{jkpmia}{m}{it}
\SetMathAlphabet{\mathfrak}{bold}{U}{jkpmia}{bx}{it}


\usetikzlibrary{arrows, positioning, shapes}    % state diagram

\geometry{                                      % page settings
  papersize={216mm,279mm},  landscape,
  top=1.525cm,  outer=1.125cm,  bottom=1.125cm, inner=1.225cm,
  includefoot,  footskip=0.5cm
}

\geometry{papersize={210mm,297mm}}              % A4 paper

% \pagecolor{black}                               % page background color
% \color{white}                                   % text default color

\renewcommand{\familydefault}{\sfdefault}       % Arial font

\lstset{
  literate=
  {á}{{\'a}}1 {à}{{\à}}1 {ã}{{\~a}}1
  {é}{{\'e}}1 {ê}{{\^e}}1
  {í}{{\'i}}1
  {ó}{{\'o}}1 {õ}{{\~o}}1
  {ú}{{\'u}}1 {ü}{{\"u}}1
  {ç}{{\c{c}}}1
}                                               % accents in code

% \delimitershortfall=-1pt                        % bigger wrapping brackets
\delimitershortfall=1pt                         % average wrapping brackets
\def\dis{\displaystyle}                         % big inline math

\makeatletter

  \def\resetMathstrut@{%                        % default \left and \right in math mode (1/2)
    \setbox\z@\hbox{%
      \mathchardef\@tempa\mathcode`\[\relax
      \def\@tempb##1"##2##3{\the\textfont"##3\char"}%
      \expandafter\@tempb\meaning\@tempa \relax
    }%
    \ht\Mathstrutbox@\ht\z@ \dp\Mathstrutbox@\dp\z@}
  \mathchardef\@tempa\mathcode`\]\relax

  \def\cantox@vector#1#2#3#4#5#6#7#8{%
    \dimen@.5\p@
    \setbox\z@\vbox{\boxmaxdepth.5\p@
    \hbox{\kern-1.2\p@\kern#1\dimen@$#7{#8}\m@th$}}%
    \ifx\canto@fil\hidewidth  \wd\z@\z@ \else \kern-#6\unitlength \fi
    \ooalign{%
      \canto@fil$\m@th \CancelColor
      \vcenter{\hbox{\dimen@#6\unitlength \kern\dimen@
        \multiply\dimen@#4\divide\dimen@#3 \vrule\@depth\dimen@\@width\z@
        \vector(#3,-#4){#5}%
      }}_{\raise-#2\dimen@\copy\z@\kern-\scriptspace}$%
      \canto@fil \cr
      \hfil \box\@tempboxa \kern\wd\z@ \hfil \cr}}
  \def\bcancelto#1#2{\let\canto@vector\cantox@vector\cancelto{#1}{#2}}

  \newlength{\normalparindent}
  \AtBeginDocument{\setlength{\normalparindent}{\parindent}}
\makeatother

\let\oldphantom\vphantom
\let\vphantom\relax
\begingroup                                     % default \left and \right in math mode (1/2)
  \catcode`(\active \xdef({\mathopen{}\left\string(\vphantom{1j}}
  \catcode`)\active \xdef){\right\string)\mathclose{}}
\endgroup
\mathcode`(="8000 \mathcode`)="8000
\let\vphantom\oldphantom

\DeclareMathSymbol{*}{\mathbin}{symbols}{"01}   % default \cdot in math mode
% #endregion

% #region commands
\newcommand*\diff{\mathop{}\!\mathrm{d}}        % differential d

\DeclareMathOperator{\dom}{dom}                 % dom function
\DeclareMathOperator{\img}{img}                 % img function
\DeclareMathOperator{\mdc}{mdc}                 % mdc function
\DeclareMathOperator{\adj}{adj}                 % adj function
\DeclareMathOperator{\proj}{proj}               % projection function

\newcommand{\Reals}{\mathds{R}}                 % Reals set
\newcommand{\Ints}{\mathds{Z}}                  % Integers set
\newcommand{\Nats}{\mathds{N}}                  % Naturals set
\newcommand{\Rats}{\mathds{Q}}                  % Rationals set
\newcommand{\Irats}{\mathds{I}}                 % Irrationals set
\newcommand{\Primes}{\mathds{P}}                % Primes set
\newcommand{\ind}{\mathds{1}}                   % indicator function

\newcommand{\given}{\,\middle|\,}               % conditional probability

\renewcommand{\complement}{\mathsf{c}}          % set complement

\DeclareRobustCommand{\Omicron}{%
  \text{\small\usefont{OMS}{cmsy}{m}{n}O}%
}                                               % big omicron
\DeclareRobustCommand{\omicron}{%
  \text{\small\usefont{OMS}{cmsy}{m}{n}o}%
}                                               % small omicron

\newcommand{\bigO}{\Omicron}                    % big O
\newcommand{\smallo}{\omicron}                  % small o

\newcommand{\defeq}{\vcentcolon=}               % definition
\newcommand{\eqdef}{=\vcentcolon}               % reverse definition

\newcommand{\cmark}{\ding{51}}                  % correct symbol
\newcommand{\xmark}{\ding{55}}                  % wrong symbol
\newcommand{\cross}{\ding{61}}                  % amortized symbol

\renewcommand{\deg}{\!\degree\:}                % math degree

\renewcommand{\binom}[2]{\l({{#1}\atop#2}\r)}   % combinations

\DeclareRobustCommand{\Chi}
  {{\mathpalette\irchi\relax}}
\newcommand{\irchi}[2]
  {\raisebox{\depth}{$#1\chi$}}                 % uppercase chi

\newcommand{\overtext}[2]
  {\overset{\mathclap{\text{#1}}}{#2}}          % text over
\newcommand{\undertext}[2]
  {\underset{\mathclap{\text{#1}}}{#2}}         % text under
\newcommand{\overmath}[2]
  {\overset{\mathclap{#1}}{#2}}                 % math over
\newcommand{\undermath}[2]
  {\underset{\mathclap{#1}}{#2}}                % math under
\newcommand{\canceltext}[2]
  {\smash{\overtext{#1}{\cancel{#2}}}}          % text cancel

\newcommand{\vghost}[1]{{%                      % artificial math size
  \delimitershortfall=-1pt
  \vphantom{
    \begingroup
    \lccode`m=`(\relax
    \lowercase\expandafter{\romannumeral#1000}%
    \lccode`m=`)\relax
    \lowercase\expandafter{\romannumeral#1000}%
    \endgroup
  }
}}

\renewcommand{\l}{\mathopen{}\left}             % \left alias
\renewcommand{\r}{\vphantom{1j}\right}          % \right alias
% \newcommand{\ls}[1]
%   {\mathopen{}\noexpand\begingroup\left#1}      % \left alias
% \newcommand{\rs}[1]
%   {\vphantom{1j}\endgroup\right#1}              % \right alias

\let\oldtextleaf\textleaf
\renewcommand{\textleaf}
  {{\fontfamily{cmr}\selectfont \oldtextleaf}}  % textleaf

\renewcommand{\qed}{\hfill$\blacksquare$}       % Black-filled qed

\newcommand{\x}[1]{\discretionary{#1}{#1}{#1}}  % correct hyphenation
\newcommand{\y}{\hspace{0pt}}                   % breackable non-space    

\newcommand{\triple}[4]{%
  \parbox{.333#4}{#1\hfill}%
  \parbox{.333#4}{\hfil#2\hfil}%
  \parbox{.333#4}{\hfill#3}%
}                                               % triple align
% #endregion

% #region customizations
\newenvironment{proof*}[1][proof*]              % better proof environment
  {\proof[#1]\vspace{0.5em}\vspace*{-\baselineskip}
  \hspace{\parindent}\leftskip=.5cm\rightskip=.5cm}
  % {\vspace*{-0.5\baselineskip}\rightskip=0cm\endproof}
  {\vspace*{-1.5\baselineskip}
  
  \rightskip=0cm\endproof}

% \let\bbordermatrix\bordermatrix
\patchcmd{\bordermatrix}{8.75}{4.75}{}{}
\patchcmd{\bordermatrix}{\left(}{\left[}{}{}    % bordermatrix with angular brackets (left)
\patchcmd{\bordermatrix}{\right)}{\right]}{}{}  % bordermatrix with angular brackets (right)
\patchcmd{\bordermatrix}
  {\begingroup}{\begingroup\openup1\jot}{}{}    % bordermatrix col height

\hypersetup{
  colorlinks,
  linkcolor={red!50!black},
  citecolor={blue!50!black},
  urlcolor={blue!80!black}
}

\lstset{
  basicstyle=\ttfamily,
  escapeinside={||},
  mathescape=true,
  extendedchars=true,
  inputencoding=utf8
}                                               % styles in listings

\DeclareFixedFont{\ttb}{T1}{txtt}{bx}{n}{8}     % for bold monofont
\DeclareFixedFont{\ttm}{T1}{txtt}{m}{n}{8}      % for normal monfont

% \definecolor{deepblue}{rgb}{0,0,0.5}
\newcommand{\pythonstyle}{\lstset{              % python code style
  xleftmargin=\dimexpr.5cm+\parindent\relax,
  language=Python,
  % basicstyle=\ttm,
  otherkeywords={self},
  % keywordstyle=\ttb\color{purple},
  keywordstyle={\bfseries},
  % emph={MyClass,__init__},
  % emphstyle=\ttb\color{deepred},
  % stringstyle=\color{deepgreen},
  % frame=tb,
  showstringspaces=false
}}

\lstnewenvironment{python}[1][]
  {\pythonstyle\lstset{#1}}{}                   % python environment

\newcommand\pythonexternal[2][]
  {{\pythonstyle\lstinputlisting[#1]{#2}}}      % external python code

\newcommand\pythoninline[1]
  {{\pythonstyle\lstinline!#1!}}                % inline python code

\renewcommand{\arraystretch}{1.2}               % table vertical padding

\renewcommand{\arraystretch}{1.2}               % table vertical padding

\lstnewenvironment{pseudocode}[1][]             % pseudocode environment
  {\lstset{
    xleftmargin=\dimexpr.5cm+\parindent\relax,
    gobble=6,
    #1,
    emph={
      if, else, elif, and, or, not,
      for, while, continue, break, return, yield, do, to,
      true, false, null
    },
    emphstyle={\bfseries}
  }}{}

\setlist[enumerate, 1]                          % default level 1 list
  {wide, label=\bfseries\arabic*., labelwidth=10pt, labelindent=0pt}
\setlist[enumerate, 2]                          % default level 2 list
  {wide, label=\bfseries(\alph*), topsep=0pt, labelwidth=10pt, labelindent=\leftskip, leftmargin=0pt}
\setlist[enumerate, 3]                          % default level 2 list
  {wide, label=\bfseries\roman*, topsep=0pt, labelwidth=10pt, labelindent=\leftskip, leftmargin=0pt}

\newenvironment{enumerate*}[1][,]               % text enumerate
  {\begin{enumerate}[
    itemindent=\leftskip+\parindent, labelindent=\leftskip+\parindent, 
    wide, topsep=0pt,
    label={\bfseries\arabic*.}, labelwidth=10pt, labelindent=\leftskip+\parindent, 
    leftmargin=\leftskip, rightmargin=\rightskip,
    #1
  ]}
  {\end{enumerate}}

\newenvironment{itemize*}[1][,]                 % text itemize
  {\begin{itemize}[
    itemindent=\leftskip+\parindent, labelindent=\leftskip+\parindent, 
    wide, topsep=0pt,
    label={\raisebox{-0.5mm}{\scalebox{1.5}{$\bullet$}}}, labelwidth=10pt, labelindent=\leftskip+\parindent, 
    leftmargin=\leftskip, rightmargin=\rightskip,
    #1
  ]}
  {\end{itemize}}
% #endregion

\begin{document}
\begin{multicols*}{2}
\setlength{\columnseprule}{0.4pt}

\begin{center}
  \textbf{\large MAC0444 --- Sistemas Baseados em Conhecimento}

  \textit{Departamento de Ciência da Computação}

  \bigskip
  \triple{\textbf{Prof.ª} Renata Wassermann}{\textbf{Lista 1}}{4 de setembro de 2019}{\columnwidth}

  \bigskip
  {\bf Aluno} Vitor Santa Rosa Gomes, 10258862, vitorssrg@usp.br

  {\bf Curso} Bacharelado em Ciência da Computação, IME-USP
\end{center}

\begin{enumerate}
  \renewcommand{\b}{\overline}
  \newcommand{\s}{\smash}
  \newcommand{\lxor}{\mathbin{\,\underline{\!\vee\!}\,}}

  \item[\textbf{1.}] Para cada uma das três sentenças abaixo, encontre uma interpretação que faça a 
  sentença falsa e as outras duas verdadeiras:
  \begin{proof*}[\unskip\nopunct]
    \begin{enumerate}
      \item $\forall x \forall y \forall z((P(x,\ y)\land P(y,\ z) \rightarrow P(x,\ z)))$
      \item $\forall x \forall y((P(x,\ y)\land P(y,\ x) \rightarrow x=y))$
      \item $\forall x \forall y((P(a,\ y)\rightarrow P(x,\ b)))$
    \end{enumerate}
    \begin{tabular}{c|ccc}
        Interpretação $\mathfrak{I} \defeq \l\langle\mathcal{D},\ \mathcal{I}\r\rangle$
      & \textbf{(a)}
      & \textbf{(b)}
      & \textbf{(c)}
      \\ \hline

        $\mathcal{D}$
      & $\mathcal{D}\defeq\l\{1,\ 2,\ 3,\ 4\r\}$
      & $\mathcal{D}\defeq\l\{1,\ 2,\ 3,\ 4\r\}$
      & $\mathcal{D}\defeq\Reals$
      \\

        $\mathcal{I}(P)$
      & $\l\{(1,\ 2),\ (2,\ 3)\r\}$
      & $\l\{(1,\ 1),\ (1,\ 2),\ (2,\ 1)\r\}$
      & $\leq$
      \\

        $\mathcal{I}(a)$
      & $1$
      & $1$
      & $1$
      \\

        $\mathcal{I}(b)$
      & $3$
      & $3$
      & $2$
      \\

        $\forall x, y, z\ ((P(x,\ y)\land P(y,\ z) \rightarrow P(x,\ z)))$
      & $x=1,\ y=2,\ z=3$ \xmark
      & \cmark
      & transitividade \cmark
      \\

        $\forall x, y\ ((P(x,\ y)\land P(y,\ x) \rightarrow x=y))$
      & \cmark
      & $x=1,\ y=2$ \xmark
      & antissimétrica \cmark
      \\

        $\forall x, y\ ((P(a,\ y)\rightarrow P(x,\ b)))$
      & $y=4$ \cmark
      & $y=4$ \cmark
      & $y=1,\ x=3$ \xmark
     \end{tabular}
  \end{proof*}

  \vfill\null\columnbreak
  \item[\textbf{2.}] Tony, Mike e John pertencem ao Clube Alpino. Todo membro do Clube Alpino que 
  não é esquiador é alpinista. Alpinistas não gostam de chuva e qualquer um que não goste de neve 
  não é esquiador. Mike não gosta de nada que Tony gosta e gosta de tudo o que Tony não gosta. Tony 
  gosta de chuva e de neve.
  \begin{proof*}[\unskip\nopunct]
    \begin{enumerate}
      \item Represente o conhecimento sobre o Clube Alpino e seus membros.
        
        Constantes: $\mathcal{C} = \l\{\text{Tony},\ \text{Mike},\ \text{John},\ 
        \text{chuva},\ \text{neve}\r\}$.

        Predicados $\mathcal{P}$:
        \begin{itemize*}
          \item $\text{alpino}(x)$: $x$ é membro do clube Alpino;
          \item $\text{esquiador}(x)$: $x$ é esquiador;
          \item $\text{alpinista}(x)$: $x$ é alpinista;
          \item $\text{gosta}(x,\ y)$: $x$ gosta de $y$.
        \end{itemize*}
        
        Base de conhecimento:
        \begin{align*}
          & \text{As constantes são distintas entre si (conhecimento sobre o domínio)}              \tag{\textbf{0}}   \\
          & \text{alpino}(\text{Tony})                                                              \tag{\textbf{1}}   \\
          & \text{alpino}(\text{Mike})                                                              \tag{\textbf{2}}   \\
          & \text{alpino}(\text{John})                                                              \tag{\textbf{3}}   \\
          & \forall x(\text{alpino}(x)\rightarrow\text{esquiador}(x)\lor\text{alpinista}(x))        \tag{\textbf{4}}   \\
          & \forall x(\text{alpinista}(x)\rightarrow\lnot\text{gosta}(x,\ \text{chuva}))            \tag{\textbf{5}}   \\
          & \forall x(\lnot\text{gosta}(x,\ \text{neve})\rightarrow\lnot\text{esquiador}(x))        \tag{\textbf{6}}   \\
          & \forall y(\text{gosta}(\text{Tony},\ y)\rightarrow\lnot\text{gosta}(\text{Mike},\ y))   \tag{\textbf{7}}   \\
          & \forall y(\lnot\text{gosta}(\text{Tony},\ y)\rightarrow\text{gosta}(\text{Mike},\ y))   \tag{\textbf{8}}   \\
          & \text{gosta}(\text{Tony},\ \text{chuva})                                                \tag{\textbf{9}}   \\
          & \text{gosta}(\text{Tony},\ \text{neve})                                                 \tag{\textbf{10}}  \\
        \end{align*}

      \item Prove semanticamente que é uma consequência lógica deste conhecimento que existe um 
      membro do Clube Alpino que é alpinista mas não esquiador.
      
        \begin{align*}
          & \lnot\text{gosta}(\text{Mike},\ \text{neve})                                            \tag{$\textbf{7}\l[x\leftarrow\text{Mike}\r]\land\textbf{10}\vDash\textbf{11}$}                 \\
          & \lnot\text{esquiador}(\text{Mike})                                                      \tag{$\textbf{6}\l[y\leftarrow\text{Mike}\r]\land\textbf{11}\vDash\textbf{12}$}                 \\
          & \text{alpinista}(\text{Mike})                                                           \tag{$\textbf{2}\land\textbf{4}\l[x\leftarrow\text{Mike}\r]\land\textbf{12}\vDash\textbf{13}$}  \\
        \end{align*}

        Existe um membro que é alpinista mas não é esquiador: Mike.

      \item Suponha que tenha sido dito apenas que Mike gosta de tudo o que Tony não gosta, mas não 
      que Mike não gosta de nada que Tony gosta. Mostre que agora a prova acima não é mais possível 
      (dê um contra-exemplo).

        Considere a interpretação canônica $\mathfrak{I}\defeq
        \l\langle\mathcal{D}\defeq\mathcal{C},\ \mathcal{I}\sim\mathcal{P}\r\rangle$, com 
        $\mathcal{I}$ obedecendo \textbf{1}, \textbf{2}, \textbf{3}, \textbf{9} e \textbf{10} e 
        mapeando $\mathcal{D}\defeq\mathcal{C}\mapsto\mathcal{C}$ de forma idêntica.

        Acrescente que Tony, Mike e John sejam esquiadores e gostem de neve.

        Assim, concluem-se \textbf{4}, \textbf{5}, \textbf{6}, \textbf{8} (contraexemplo).

      \item Use resolução com extração de resposta para descobrir quem é o membro do Clube Alpino 
      que é alpinista mas não esquiador.

        Base de conhecimento com a pergunta:
        \begin{align*}
          & \text{As constantes são distintas entre si (conhecimento sobre o domínio)}              \tag{\textbf{0}}  \\
          & \text{alpino}(\text{Tony})                                                              \tag{\textbf{1}}  \\
          & \text{alpino}(\text{Mike})                                                              \tag{\textbf{2}}  \\
          & \text{alpino}(\text{John})                                                              \tag{\textbf{3}}  \\
          & \lnot\text{alpino}(x)\lor\text{esquiador}(x)\lor\text{alpinista}(x)                     \tag{\textbf{4}}  \\
          & \lnot\text{alpinista}(x)\lor\lnot\text{gosta}(x,\ \text{chuva})                         \tag{\textbf{5}}  \\
          & \text{gosta}(x,\ \text{neve})\lor\lnot\text{esquiador}(x)                               \tag{\textbf{6}}  \\
          & \lnot\text{gosta}(\text{Tony},\ y)\lor\lnot\text{gosta}(\text{Mike},\ y)                \tag{\textbf{7}}  \\
          & \text{gosta}(\text{Tony},\ y)\lor\text{gosta}(\text{Mike},\ y)                          \tag{\textbf{8}}  \\
          & \text{gosta}(\text{Tony},\ \text{chuva})                                                \tag{\textbf{9}}  \\
          & \text{gosta}(\text{Tony},\ \text{neve})                                                 \tag{\textbf{10}} \\
          & \lnot\text{alpino}(x)\lor\lnot\text{alpinista}(x)
            \lor\text{esquiador}(x)\lor\text{resposta}(x)                                           \tag{\textbf{11}} \\
        \end{align*}

        Resolução:
        \begin{align*}
          & \lnot\text{gosta}(\text{Mike},\ \text{neve})                                            \tag{$\textbf{7}\l[y\leftarrow\text{Mike}\r]\land\textbf{10}\vDash\textbf{11}$} \\
          & \lnot\text{esquiador}(\text{Mike})                                                      \tag{$\textbf{6}\l[x\leftarrow\text{Mike}\r]\land\textbf{11}\vDash\textbf{12}$} \\
          & \text{esquiador}(\text{Mike})\lor\text{alpinista}(\text{Mike})                          \tag{$\textbf{2}\land\textbf{4}\l[x\leftarrow\text{Mike}\r]\vDash\textbf{13}$}  \\
          & \text{alpinista}(\text{Mike})                                                           \tag{$\textbf{12}\land\textbf{13}\vDash\textbf{14}$}                            \\
          & \lnot\text{alpinista}(\text{Mike})\lor\text{esquiador}(\text{Mike})
            \lor\text{resposta}(\text{Mike})                                                        \tag{$\textbf{3}\land\textbf{11}\l[x\leftarrow\text{Mike}\r]\vDash\textbf{15}$} \\
          & \lnot\text{alpinista}(\text{Mike})\lor\text{resposta}(\text{Mike})                      \tag{$\textbf{12}\land\textbf{15}\vDash\textbf{16}$}                            \\
          & \text{resposta}(\text{Mike})                                                            \tag{$\textbf{14}\land\textbf{15}\vDash\textbf{17}$}                            \\
        \end{align*}
    \end{enumerate}
  \end{proof*}
\end{enumerate}
\end{multicols*}
\end{document}